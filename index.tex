% Created 2024-11-26 mar 09:55
% Intended LaTeX compiler: pdflatex
\documentclass[11pt]{article}
\usepackage[utf8]{inputenc}
\usepackage[T1]{fontenc}
\usepackage{graphicx}
\usepackage{longtable}
\usepackage{wrapfig}
\usepackage{rotating}
\usepackage[normalem]{ulem}
\usepackage{amsmath}
\usepackage{amssymb}
\usepackage{capt-of}
\usepackage{hyperref}
\usepackage{minted}
\usepackage[spanish]{inputenc}
\author{likcos}
\date{\today}
\title{Proyectos IA Final}
\hypersetup{
 pdfauthor={likcos},
 pdftitle={Proyectos IA Final},
 pdfkeywords={},
 pdfsubject={},
 pdfcreator={Emacs 28.2 (Org mode 9.5.5)}, 
 pdflang={Spanish}}
\begin{document}

\maketitle
\tableofcontents

\section*{Actividad 1}
\label{sec:orgb8278e5}
\subsection*{Implementación del Algoritmo A*}
\label{sec:org6234bb5}
Desarrollar el algoritmo A* utilizando el cascarón proporcionado en el apartado de \href{https://ealcaraz85.github.io/IA.io/\#org0d76d38}{pygames}. 

\subsubsection*{Requisitos}
\label{sec:org0844da0}
\begin{itemize}
\item La solución debe ser óptima.
\item Indicar la lista cerrada al final del proceso.
\end{itemize}

\section*{Actividad 2}
\label{sec:org0944a6b}
\subsection*{Solución basada en Árboles de Decisión y Redes Neuronales Multicapa}
\label{sec:orgf81b47b}
A partir de la adaptación del juego de Phaser a Python:

\subsubsection*{Tareas}
\label{sec:orgc9d3410}
\begin{enumerate}
\item Implementar la solución utilizando árboles de decisión.
\item Implementar la solución utilizando redes neuronales multicapa.
\end{enumerate}

\subsubsection*{Objetivo}
\label{sec:org88610ee}
El jugador debe esquivar una pelota saltando.

\subsubsection*{Ejemplo de Redes Neuronales Multicapa en Python}
\label{sec:org2803648}
\begin{minted}[]{python}
import numpy as np
from tensorflow.keras.models import Sequential
from tensorflow.keras.layers import Dense
from sklearn.model_selection import train_test_split

# Generar datos artificiales
np.random.seed(0)
X = np.random.rand(1000, 2)  # 1000 puntos con 2 características cada uno
y = (X[:, 0] + X[:, 1] > 1).astype(int)  # Etiqueta 1 si la suma de las características > 1, de lo contrario 0

# Dividir los datos en conjuntos de entrenamiento y prueba
X_train, X_test, y_train, y_test = train_test_split(X, y, test_size=0.2, random_state=42)

# Crear el modelo de red neuronal multicapa
model = Sequential([
    Dense(4, input_dim=2, activation='relu'),  # Capa oculta con 4 neuronas y activación ReLU
    Dense(1, activation='sigmoid')            # Capa de salida con 1 neurona y activación sigmoide
])

# Compilar el modelo
model.compile(optimizer='adam',
              loss='binary_crossentropy',
              metrics=['accuracy'])

# Entrenar el modelo
model.fit(X_train, y_train, epochs=20, batch_size=32, verbose=1)

# Evaluar el modelo en el conjunto de prueba
loss, accuracy = model.evaluate(X_test, y_test, verbose=0)
print(f"\nPrecisión en el conjunto de prueba: {accuracy:.2f}")

# Probar con un nuevo dato
nuevo_dato = np.array([[0.8, 0.3]])  # Ejemplo con características específicas
prediccion = model.predict(nuevo_dato)
print(f"Predicción para {nuevo_dato}: {prediccion[0][0]:.2f}")
\end{minted}

\subsubsection*{Descripción del ejemplo}
\label{sec:org83bcc4a}
\begin{enumerate}
\item \textbf{\textbf{Datos artificiales}}:
\begin{itemize}
\item Se generan 1000 puntos en 2D, cada uno con dos características.
\item Las etiquetas se asignan como 1 si la suma de las características es mayor que 1, de lo contrario 0.
\end{itemize}
\item \textbf{\textbf{Modelo de red neuronal}}:
\begin{itemize}
\item Capa oculta: Tiene 4 neuronas con activación ReLU.
\item Capa de salida: Tiene 1 neurona con activación sigmoide, lo que da como resultado una probabilidad entre 0 y 1.
\end{itemize}
\item \textbf{\textbf{Compilación}}:
\begin{itemize}
\item Función de pérdida: `binary\textsubscript{crossentropy}`, adecuada para clasificación binaria.
\item Métrica: `accuracy` (precisión).
\end{itemize}
\item \textbf{\textbf{Predicción}}:
\begin{itemize}
\item Se pasa un nuevo punto al modelo y se obtiene una probabilidad. Si está cerca de 1, la salida es 1; si está cerca de 0, la salida es 0.
\end{itemize}
\end{enumerate}

\section*{Actividad 3}
\label{sec:org2ef69c3}
\subsection*{Identificación de modelos de autos con CNN}
\label{sec:orgdf7f651}
Utilizando el archivo `CNNriesgo` que se encuentra en la carpeta de Dropbox:

\subsubsection*{Actividad}
\label{sec:org4bcefc5}
Ajustar el dataset para identificar cinco modelos diferentes de autos.

\subsubsection*{Evaluación}
\label{sec:org90ecfb6}
\begin{itemize}
\item Herramientas utilizadas para la creación del dataset.
\item Precisión con la que se detectan los modelos.
\end{itemize}

\section*{Actividad 4}
\label{sec:org1bf7a16}
\subsection*{Fundamentación sobre la Reforma al Poder Judicial y Organismos Autónomos}
\label{sec:org33fccee}
Utilizando algoritmos de inteligencia artificial, responder las siguientes preguntas y fundamentar si estás a favor o en contra de la reforma al poder judicial y a los organismos autónomos.

\subsubsection*{Preguntas para la Ley del Poder Judicial}
\label{sec:org0c9860f}
\begin{enumerate}
\item ¿El diagnóstico de la ley al poder judicial es conocido y qué estudios expertos se tuvieron en cuenta?
\item ¿Por qué la reforma no incluyó a las fiscalías y a la defensoría, limitándose solo al poder judicial?
\item ¿Qué medidas concretas se implementarán para evitar la captación del crimen organizado y la violencia en el contexto electoral?
\item ¿Cómo garantizar que juristas probos y honestos se animen a competir públicamente frente a los riesgos de la violencia?
\item ¿Cómo se conforman los comités de postulación?
\item ¿Cómo asegurar la carrera judicial?
\item ¿Cómo compatibilizar la incorporación de medidas para preservar la identidad de los jueces (conocidos en el sistema interamericano como "jueces sin rostro") con los estándares internacionales?
\item ¿Cómo impactará el costo económico de esta reforma en la promoción y el acceso a la justicia?
\end{enumerate}

\subsubsection*{Preguntas para la Ley de Organismos Autónomos}
\label{sec:orgb242f28}
\begin{enumerate}
\item ¿Es constitucional esta ley, considerando que algunos organismos autónomos están establecidos en la Constitución?
\item ¿Cómo afectaría la eliminación de estos organismos a la transparencia y rendición de cuentas del gobierno?
\item ¿Qué funciones críticas podrían perder independencia y control al pasar al poder ejecutivo u otras instituciones?
\item ¿Existen alternativas para mejorar la eficiencia de los organismos autónomos sin eliminarlos?
\item ¿Qué sectores de la sociedad civil y grupos de interés se verían afectados por la desaparición de estos organismos?
\end{enumerate}

\subsubsection*{Puntos a evaluar}
\label{sec:orgf0e3274}
\begin{itemize}
\item Algoritmos utilizados.
\item Herramientas generadas para el análisis.
\item Datos utilizados para la fundamentación.
\item Proceso de análisis.
\end{itemize}
\end{document}